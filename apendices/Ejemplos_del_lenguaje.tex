\chapter{Ejemplos del lenguaje}\label{ejemplos_lenguaje}

Estos ejemplos quieren ilustrar el uso de la herramienta en las distintas áreas de la 
computación.

\section{Caso 1: Base de datos}
\label{ejemplo:bd}
En este caso se quiere crear objetos que representan a 5 personas para una base 
de datos cuya única restricción es que sus cédulas aparezcan de mayor a menor. 
Los datos que tiene cada persona pueden ser:

\begin{itemize}
\item{La cédula que va desde 1000 a 1100.}
\item{El nombre que puede ser alguno de estos: Juan, Pedro, Marco, Jose, Isaac, Tony, Alexis, Erick, Hancel, Alfredo o Carlos.}
\item{Y finalmente el nombre de la bebida que le gusta, que puede ser alguna de estas: Agua, Té, Pepsi, CocaCola o Nestea.}
\end{itemize}

\begin{lstlisting}[mathescape]
salida personas {
 descripcion {
  persona p1;
  persona p2;
  persona p3;
  persona p4;
  persona p5;
 }
 restriccion {
  p1.cedula > p2.cedula;
  p2.cedula > p3.cedula;
  p3.cedula > p4.cedula;
  p4.cedula > p5.cedula;
 }
}

aux persona {
 descripcion {
  int (1000, 1100) cedula;
  Str nombre =$\sim$ ["Juan" | "Pedro" | "Jose" | "Isaac" | "Tony" | 
            "Alexis" | "Erick" | "Hancel" | "Alfredo" | "Carlos" | "Marco"];
  Str bebida =$\sim$ ["Agua" | "Te" | "Pepsi" | "CocaCola" | "Nestea"];
 }
}
\end{lstlisting}

\section{Caso 2: Computación gráfica}
\label{ejemplo:cg}
Para este ejemplo se quiere crear escenarios con casas aleatorias y los datos de estas son:

\begin{itemize}
\item{Número de pisos que pueden ser entre 1 y 3.}
\item{Altura de los pisos que debe ser un flotante entre 2 y 3.}
\item{Ancho y profundidad de la casa que esta entre 10 y 20.}
\item{Un booleano que indique que si la casa tiene chimenea o no.}
\end{itemize}

\begin{lstlisting}[mathescape]
salida casa {
 descripcion { 
  int (1,3) pisos; 
  float (2,3) alturaPiso; 
  float (10,20) ancho; 
  float (10,20) profundidad; 
  bool chimenea; 
 } 
}
\end{lstlisting}

\section{Caso 3: HTML}
\label{ejemplo:html}
Para este caso lo que se quiere es probar distintas alturas de \emph{header}, \emph{content}
y \emph{footer} dentro de una página web, la única retricción que que tienen en 
común es que tengan el mismo ancho y los valores que admite estan entre 750 y 800.
Las restricciones en particular cada una de estas partes son:

\begin{itemize}
\item{\emph{Header} tiene un alto que puede ser alguno de estos valores 100, 150, 200, 250 y 300.}
\item{\emph{Content} tiene un alto que puede ser entre 500 y 800.}
\item{\emph{Footer} tiene un alto que puede ser entre 50 y 100.}
\end{itemize}

\begin{lstlisting}[mathescape]
salida html { 
 descripcion { 
  header  parte1;
  content parte2; 
  footer  parte3; 
 } 
 restriccion { 
  parte1.width == parte2.width; 
  parte1.width == parte3.width; 
 } 
} 
aux header { 
 descripcion { 
  int height $=\sim$ [100 $\mid$ 150 $\mid$ 200 $\mid$ 250 $\mid$ 300]; 
  int (750,800) width; 
 } 
} 
aux content { 
 descripcion { 
  int (500,800) height; 
  int (750,800) width; 
 } 
}
aux footer {
 descripcion { 
  int (50,100) height; 
  int (750,800) width; 
 } 
}
\end{lstlisting}
