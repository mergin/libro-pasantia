\chapter{Retos Enfrentados y Logros Adicionales}\label{chapter:Retos y logros}

Este capítulo muestra aquellos retos que se presentaron a lo largo de la pasantía; así como la descripción de algunos logros adicionales que no estaban en el plan inicial.

\section{Retos enfrentados} \label{sect:Retos}
Durante el desarrollo del proyecto de pasantía surgieron una serie de retos considerables. El primer reto en presentarse fue el trabajar con una herramienta aún nueva, con la que solo un pasante había trabajado anteriormente, y que todavía está en su infancia en su desarrollo. Esto significó defectos y errores en la plataforma que dificultaran el aprendizaje y desarrolo además de poco soporte y documentación comparado con las plataformas nativas móviles.

Luego, la interfaz proporcionada por el departamento de diseño para la aplicación móvil constituyó un gran reto ya que, aunque la aplicación es simple, la interfaz de diseño minimalista contiene muchas particularidades que no son provistas por la plataforma Ionic. Por tanto se tuvieron que modificar estas y hacer trucos de programación para lograr los requerimientos visuales y funcionales de la interfaz. Hay dos casos en particular que cabe resaltar: los botones múltiples del menú principal y los botones filtros de la página buzón.

Los botones del menú principal fueron un reto en el sentido en que se requería una forma particular de los botones, no el estándar rectangular que provee Ionic, así que se tuvo que improvisar botones invisibles, parcialmnete solapados entre si, que estuvieran libre del contexto del resto de la página para colocarlos debajo de la imágen de botones provistas por el departamento de diseño para que funcionaran como se requería. Los botones filtros significaron tener que modificar la funcionalidad normal las cabeceras de página y de los botones provista por Ionic. Se tuvo que modificar la acción de los botones para que estos se quedaran presionados mientras ejercían su función como filtros, pero que a la vez solamente se permitiera uno presionado a la vez. Estos dos retos fueron superados gracias a la manipulación de las funcionalidades de AngularJS sobre HTML y CSS de la aplicación, con ideas provistas por la comunidad de StackOverflow.

Otro reto importante fue el mantener un buen desempeño de la aplicación, ya que posiblemente tenga que manejar grandes cantidades de mensajes. Por lo tanto se hizo uso de las herramientas que AngularJS ofrece para que la manipulación de mensajes (apertura, eliminación y guardado) sea lo más eficiente posible.


Posiblemente el reto más importante, no por su dificultad sino por lo crítico que era para la funcionalidad de la aplicación, era el uso del plugin para la recepción de notificaciones push en Cordova. Primero no se tenía mucho soporte ya que era un plugin nuevo, hecho por terceros ajenos a la plataforma Ionic, y se requirió mucho tiempo y ensayo y error para hacerlo funcionar. Después se descubrió un fallo: las notificaciones push provenientes de la plataforma Synergy Push no eran recibidas cuando la aplicación estaba en segundo plano. Se detectó el orígen del problema en un fallo del dieño del código fuente del plugin de Cordova, por lo que se tuvo que modificar el código fuente para solventarlo.


\section{Logros adicionales} \label{sect:Logros}
Durante el desarrollo del proyecto de pasantía se alcanzaron una serie de logros adicionales que no se encontraban estipulados en el plan de trabajo inicial. Entre ellos se encuentra la realización las maquetas iniciales que normalmente corresponderían al equipo de diseño o equipo encargado de planificar el producto, pero fue delegada al pasante que entregó al dueño del producto para su aprobación y luego al equipo de diseño para la fabricación de la interfaz definitiva. El dueño del producto especificó las funcionalidades que debía tener la aplicación y a partir de estos el pasante creó los casos de uso y luego las maquetas en base a esos casos. 

Para que el sistema de tipos de mensaje (Promoción, Aprobados, Alerta y ATM) y su clasificación (Respuesta Simple, Respuesta de dos Opciones, o WebView) funcionara, se tuvo que crear un estándar de identificación para que cada mensaje proveniente de Synergy Push estuviera debidamente identificado y la aplicación supiera identificarlos y tratarlos. Esto fue un logro adicional ya que la empresa no había decidido todavía como tratar a los mensajes. Solo estaba hecha la plataforma Synergy Push que trata mensajes genéricos y el concepto de la aplicación. Durante este trayecto se descubrió un error en la implementación de Synergy Push del cual la empresa no estaba al tanto, y fue informada para su futura corrección. Mientras estas se llevaban a cargo se implementó una solución provisional dentro de la aplicación que solventara el error contenido en los mensajes, haciendo uso de expresiones regulares.

Otro defecto encontrado de la plataforma Synergy Push es que actualmente envía los mensajes sin fecha, por lo que fue necesario agregarle a Notificaciones+ la funcionalidad de agregarle fecha a los mensajes cuando son recibidos.

La modificación del código fuente del plugin de recepción de notificaciones push de Cordova se cuenta como un logro adicional. El defecto del diseño de este plugin también fue reportado en su repositorio y en la comunidad de StackOverflow.

Finalmente, otro logro adicional fue la localización de otro error, pero esta vez es en la plataforma Ionic, que fue reportado a los creadores. La plataforma Ionic posee animaciones cuando manipula elementos en una lista, y uno de estas animaciones desencadena un error fatal cuando corre bajo la plataforma iOS 7. Para solventarlo se eliminó la animación exclusivamente en esa plataforma.
