\setcounter{page}{4}
\begin{center}

{\bfseries DESARROLLO APLICACION NOTIFICACIONES+\\}
\bigskip
Por: \\ David Enrique Prieto Melendez\\
\bigskip
\bigskip
{\bf RESUMEN} \pdfbookmark[0]{Resumen}{resumen} % Sets a PDF bookmark for the dedication
\end{center}	

El presente informe describe las actividades realizadas durante el proyecto de pasantía larga, el cual consistió en diseñar y desarrollar un prototipo funcional de la aplicación web de \textit{Notificaciones+} y un prototipo funcional de la aplicación móvil del mismo. Este conjunto de aplicaciones están orientadas a los clientes de bancos, que tengan un dispositivo con la capacidad de desplegar un navegador web, o un dispositivo que funcione bajo las plataformas Android e iOS. Los prototipos funcionales de \textit{Notificaciones+} permiten la recepción de mensajes clasificados de acuerdo a su tipo, que un banco desee mandar a sus clientes. También se permite a los clientes mandar una respuesta al banco según el tipo específico del mensaje, en caso de que el banco requiera una respuesta por parte de sus clientes para una operación.


Para lograr el conjunto de objetivos de \textit{Notificaciones+} fue necesaria la implementación de la lógica de las aplicaciones para el manejo de la recepción y gestión de mensajes provenientes de la plataforma \textit{Synergy Push}. Desde el punto de vista tecnológico se hizo uso de herramientas y tecnologías como JavaScript, AngularJS, Ionic Framework, Node.js. Los resultados de esta pasantía, realizados de acuerdo a la Metodología de Synergy-GB y documentados con artefactos de desarrollo de sistemas implantados en la empresa, incluyen el diseño y desarrollo de \textit{Notificaciones+}, así como la integración adecuado con los servicios web y push de la plataforma \textit{Synergy Push}.

\bigskip
\noindent
PALABRAS CLAVES: IONIC, MULTIPLATAFORMA, PUSH, MENSAJERÍA, WEB.
